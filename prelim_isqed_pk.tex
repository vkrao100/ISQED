\section{Preliminaries: Notation and Background}\label{sec:prelim}
This section reviews some basic concepts from symbolic computer
algebra and associated algorithms that are utilized in this paper. 

{\it Finite Field and Primitive Polynomial:}
%Let $\B$ denote the Boolean domain, and $\neg, \wedge,\vee, \oplus$
%denote the  NOT, AND, OR and XOR operators, respectively.
Let $\F_2=\{0,1\}$ be the field of 2 elements, and let $\F_q =
\F_{2^n}$ denote the finite 
field of $q=2^n$ elements, for a given $n$. $\Fn$ is the
$n$-dimensional extension of $\F_2$, and it is constructed as $\Fn =
\F_2[X]\pmod{ P_n(X)}$. Here $P_n(X) \in \F_2[X]$ is a {\it primitive
  polynomial}, i.e. a degree-$n$ polynomial, irreducible in
$\F_2$, with a root $\gamma$ ($P_n(\gamma)=0$). 
Consequently, $\gamma$ is called a primitive element (PE) of $\Fn$ and
it generates the cyclic group: $\Fn^*=\{1=\gamma^{2^n - 1}, \gamma,
\gamma^2,\dots,\gamma^{2^n - 2}\}$. Note: $\forall \gamma\in \Fkn,
\gamma^{2^n-1}=1$, thus $\gamma^{2^n}=\gamma$.

{\it Minimal Polynomial of $\gamma$:} If $\gamma \in \Fn$ is a root
of $P_n(X)$, then $\gamma^{2^l},l\geq 0$, is also a root of
$P_n(X)$. Elements $\gamma^{2^l}$ are conjugates of each other. Let
$e>0$ be the smallest integer such that $\gamma^{2^e} = \gamma$. Then
$P_n(X)=\prod_{i=0}^{i=e-1}(X+\gamma^{2^i})$ is called the {\it
  minimal polynomial} of $\gamma$. If $\gamma$ is also a PE of $\Fn$,
then the minimal polynomial of $\gamma$ is also a primitive
polynomial, and has degree $=n$. Hence, $\Fn$ is also called the
$\gamma$-extension of $\F_2$, also denoted as $\F_2(\gamma)$. 
An element $A \in \Fn$ can be written as $A = a_0 + a_1\cdot \gamma +
\dots + a_{n-1}\cdot\gamma^{n-1}$, where $a_0,\dots,a_{n-1} \in
\F_2$. 
In $\Fq$, the addition ($''+''$) and multiplication
($''\cdot''$) operations are performed in the base field $\F_2$ and
reduced modulo the corresponding primitive polynomial $P_n(X)$. 
For $n, k \in \Z_{> 0}$, if $n$ divides $k$ ($n ~|~ k$), then $\Fn \subset
\Fkk$. 
Thus, $\F_2 \subset \Fkk, \forall k>1$. 
The fields $\Fkn$ have
characteristic 2, and therefore $-1 = +1$ in $\Fkn$. 


\begin{Example}
{\it
Consider the fields $\F_{2^4}$ and $\F_{2^2}$. Let $\F_{2^4}$ be
constructed as $\F_2[X] \pmod{ P_4(X)=X^4+X^3+1}$, with $P_4(\gamma)=0$. 
Let $\F_{2^2} = \F_2[X]\pmod{
  P_2(X)=X^2+X+1}$, with $P_2(\alpha)=0$. Since $2 ~|~ 4$, $\F_{2^2}
\subset \F_{2^4}$. We demonstrate this field containment in terms of
their PEs. 

Compute $\gamma^{2^l}$ for $l>0$: for
$l=1,\gamma^{2^l}=\gamma^2$; $l=2, \gamma^{2^l}=\gamma^4$; $l=3,
\gamma^{2^l}=\gamma^8$; $l=4, \gamma^{2^l}=\gamma^{16}=\gamma$. Since
$\gamma^{16}=\gamma$ (the elements repeat), we have that $\{\gamma,
\gamma^2, \gamma^4, \gamma^8\}$ are conjugates. Thus the minimal
polynomial of $\gamma$ can be reconstructed as
$(X+\gamma)(X+\gamma^2)(X+\gamma^4)(X+\gamma^8)$, which simplifies to
$X^4+X^3+1 = P_4(X)$
itself. Similarly, $\{\alpha,\alpha^2\}\in \F_{2^2}$ are conjugates, and
$(X+\alpha)(X+\alpha^2)=P_2(X)$.

Now consider the element $\gamma^5 \in \F_{2^4}$. Then $\gamma^{10}$
is its only conjugate, and $(X+\gamma^5)(X+\gamma^{10}) =
X^2+X+1=P_l(X)$. In other words, $\{0, 1, \alpha = \gamma^5, \alpha^2 =
\gamma^{10}\}$ are the 4 elements of $\F_{2^2}$, also contained in
$\F_{2^4}$; implying that $\F_{2^2} \subset \F_{2^4}$.
}
\end{Example}

% To model functions over $n$-bit vector operands, we
% use $q = 2^n$, i.e. the finite field $\Fkn$ of $2^n$ elements. 
Let $R=\F_q[x_1,\dots,x_d]$ be the  polynomial ring in variables
$x_1,\dots,x_d$ with coefficients in $\F_q$. A polynomial $f \in R$ is 
written as a finite sum of terms  $f = c_1 M_1 +  \dots +
c_p M_p$, where $c_i$'s are the coefficients and $M_i$'s
the monomials, and $c_i M_i$'s terms of $f$.
%, i.e. power products of the type $x_1^{e_{1}}\cdot x_2^{e_{2}}\cdots
%x_d^{e_{d}}$,  $e_j \in \Z_{\geq  0}$. To systematically manipulate
%the polynomials,
To systematically manipulate the polynomials, a term order $>$
is imposed on all polynomials of $R$, such that subject to $>$,
$c_1M_1 > c_2 M_2 > \dots > c_p M_p$. Then $lt(f) = c_1 M_1$ denotes
the leading term of $f$. 
% This work employs {\it lexicographic} (lex) term orders. 

%%%%%%%%%%%%%%%%%%%% PK
%% A circuit with $n$-bit operands is modeled as a polynomial function in
%% $\Fn$, where variables $x_1,\dots,x_d$ denote the nets of the circuit.
%% %%%% Move to verification
%% Logic gates of the circuit can be modeled with polynomials in
%% $\F_2 \subset \Fn$, with the mapping $\B \mapsto \F_2$:
%% %% As $\Fkk \supset \F_2$, these polynomials can also be
%% %% construed as polynomials in $\Fkk$.
%% %% given as: 
%% \begin{equation}
%% \label{bool2poly}
%% \begin{split}
%% z ~ =  ~ \neg a ~ \rightarrow ~ z+a+1 & \pmod 2  \\
%% z ~ =  ~ a \wedge b ~ \rightarrow ~ z+a \cdot b & \pmod 2\\
%% z ~ =  ~ a \vee b ~ \rightarrow ~ z+a+b+a \cdot b & \pmod 2 \\
%% z ~ =  ~ a \oplus b ~ \rightarrow ~ z+a+b & \pmod 2 
%% \end{split}
%% \end{equation}

{\it Polynomial Reduction:} Let $F=\{f_1,\dots,f_s\}$ be
a set of polynomials in $R$ and $f\in R$ be 
another polynomial. Then $f\xrightarrow{F}_+r$ denotes the {\it
  reduction} of $f$ modulo $F$ resulting in remainder $r$, which is
obtained by iteratively canceling terms in $f$ by $lt(f_j), f_j\in F$,
via polynomial division (cf. Alg. 1.5.1~\cite{gb_book}). The
remainder $r$ is said to be {\it reduced} such that no term in $r$ is
divisible by the leading term of any $f_j \in F$.

%Algorithm~\ref{algo:mv_reduce}
% term of any polynomial $f_j$ in $F$.
% Along with the remainder $r$, the algorithm also returns
% the set of quotients $\{u_1,\dots,u_s\}$ of division of $f$ by
% $\{f_1,\dots,f_s\}$, respectively, such that $f = u_1\cdot
% f_1+\dots+u_s\cdot f_s + r$.}

% {\small
% \begin{algorithm}[hbt]
%  \caption{Multivariate Reduction of $f$ by $F=\{f_1,\dots,f_s\}$}
%  \label{algo:mv_reduce}
%  \begin{algorithmic}[1]
%  % \Procedure{$multi\_variate\_division$}{$f, f_1, \dots, f_s \in \F[x_1, \dots, x_d], f_i\neq 0$}
%  \Procedure{$multi\_var\_division$}{$f,\{f_1,\dots,f_s\},f_j\neq0$}
%  % \ENSURE $u_1,\dots, u_s, r$ s.t. $f = \sum f_i u_i+r$ where $r$ is
%  % reduced w.r.t. $F = \{f_1,\dots, f_s\}$ and max($lp(u_1)lp(f_1), \dots, lp(u_s)lp(f_s), lp(r)$) = $lp(f)$
%  \State $u_j \gets 0; ~r \gets 0, ~h \gets f $ 
%  \While {  $h \neq 0$ }
%  \If{ $\exists j$ s.t. $lm(f_j) ~|~ lm(h)$}
%  \State choose $j$ least s.t. $lm(f_j) ~|~ lm(h)$
%  \State $u_j = u_j + \frac{lt(h)}{lt(f_j)}$
%  \State $h = h - \frac{lt(h)}{lt(f_j)} f_j$
%  \Else
%  \State $r = r+ lt(h)$
%  \State $h = h - lt(h)$
%  \EndIf
%  \EndWhile
%  \State \Return $(\{u_1,\dots,u_s\} , r)$
%  \EndProcedure
%  \end{algorithmic}
%  \end{algorithm}
% }

%% The algorithm initializes $h$ with the polynomial $f$ and cancels its
%% leading term by some  polynomial $f_j$. If the leading term $lt(h)$
%% cannot be canceled by any $lt(f_j)$, then it is added to the  final
%% remainder $r$ and the process is repeated until all the terms in $h$
%% are analyzed.  
%% We represent the given circuit by way of a set of polynomials
%% $F=\{f_1,\dots,f_s\}$, and for verification and rectification, we
%% analyze the solutions to the polynomial equations $f_1=\dots=f_s=0$. 
%% For this purpose, we
%% consider the {\it ideal} generated by the polynomials, and their {\it
%%   variety.} 

%\begin{Definition}$\bf{\left[Ideal\right]}$
{\it Ideals:} A set of polynomials 
$F=\{f_1,\dots,f_s\}$ from $R$, generates the {\bf ideal} $J = \langle
F \rangle \subseteq R$, defined as $J = \langle f_1,\dots,$ $ f_s \rangle = \{
h_1\cdot f_1 + \dots+h_s\cdot f_s:  h_1,\dots,h_s\in R\}$. 
%% \vspace{-0.1in}
%% \begin{small}
%% \begin{equation}
%% J = \langle f_1, \dots, f_s \rangle = \{ h_1\cdot f_1 + \dots+h_s\cdot
%% f_s:  h_1,\dots,h_s\in R\}
%% \end{equation}
%% \end{small}
%% \vspace{-0.1in}
Polynomials $f_1,\dots,f_s$ form the basis or generators of ideal $J$.
%\end{Definition}

{\it Varieties:} Let $\bm{a} = (a_1,\dots,a_d) \in \Fq^d$ be a point in the affine
space, and $f$ a polynomial in $R$. If $f(\bm{a}) = 0$, we say
that $f$ {\it vanishes} on $\bm{a}$. We have to
analyze the {\it set of all common zeros} of the polynomials of $F =
\{f_1,\dots,f_s\}$  that lie within the field $\Fq$ -- i.e. the set of
all points $\bm{a} \in \Fq^d$ such that
$f_1(\bm{a})=\dots=f_s(\bm{a})=0$. This zero set is called the {\bf
  variety}, which depends not just on the given set of polynomials in
$F$, but rather on the ideal generated by them. We denote the variety as
$V(J)$, where: $ V(J)= V_{\Fq}(J) = V_{\Fq}(f_1,\dots,f_s) = \{\bm{a}
\in \Fq^d: \forall f \in J, f(\bm{a}) = 0\}.$


Given two ideals $J_1 = \langle f_1,\dots,f_s\rangle, J_2=\langle
h_1,\dots,h_r\rangle$, we denote their {\bf sum} $J_1 + J_2 = \langle
f_1,\dots,f_s,h_1\dots,h_r\rangle$, and their {\bf product} $J_1\cdot J_2 =
\langle f_i\cdot h_j: 1\leq i\leq s, 1\leq j\leq r\rangle$.
%, and the {\bf ideal quotient (colon ideal)} of $J_1$ by $J_2$,
%$J_1:J_2 = \{f \in R \ |\ f\cdot g \in J_1, \forall g \in
%J_2\}$.
Ideals and varieties are dual concepts: $V(J_1 + J_2) = V(J_1) \cap
V(J_2)$, and $V(J_1\cdot J_2) = V(J_1) \cup V(J_2)$.
%, and $V(J_1:J_2)=V(J_1)-V(J_2)$. 
%Moreover, if $J_1 \subseteq J_2$ then $V(J_1)\supseteq V(J_2)$.

% \begin{Definition}
% \label{def:quo}
% ({\it Quotient of Ideals}) If $J_1$ and $J_2$ are ideals in a ring $R$,
% then $J_1:J_2$ is the set 
% %  \begin{equation}
%   $\{f \in R \ |\ f\cdot g \in J_1, \forall g \in J_2\}$ %\nonumber
% %  \end{equation}
% and is called the {\bf ideal quotient} of $J_1$ by $J_2$, also called
% the {\bf colon ideal}.
% \end{Definition}
% Given the generators of $J_1$ and $J_2$, the generators of $J_1:J_2$ can
% also be computed using \Grobner bases with elimination (lex) term
% orders. We refer the reader to Section 2.3 in \cite{ideals:book} for
% further details.
% Varieties can be different when restricted to the given field $\Fq$
% or considered over its algebraic closure $\Fqbar$. We will generally
% drop the subscript when considering varieties over $\Fq$ and
% denote $V(J)$ to imply $V_{\Fq}(J)$. The subscripts will be used,
% however, to avoid any ambiguities, e.g. when comparing $V_{\Fq}(J)$
% against the one over the closure $V_{\Fqbar}(J)$. 


{\it \Grobner Bases:} An ideal $J$ may have many different sets of 
generators/bases. %such that their varieties are the same.
A \Grobner basis (GB) is one such generating set $G=\{g_1, \dots,
g_t\}$ with special properties.
%  that help to solve
% many polynomial decision and quantification problems. 
 % that is a canonical representation of the ideal. 

\begin{Definition}
\label{def:gb}
{[\Grobner Basis]}~\cite{gb_book}: 
For a monomial ordering $>$, a set of non-zero polynomials $G =
\{g_1,\cdots,g_t\}$ contained in an ideal $J$, is called a
GB of $J$ $\iff$
$\forall f \in J$, $f\xrightarrow{g_1,..,g_t}_+0$. 
%$i \in \{1,\cdots, t\}$ 
\end{Definition}
The GB $G$ for an ideal $J$ can be computed using the Buchberger's
algorithm  %\cite{buchberger_thesis}. 
(cf. Alg. 1.7.1 in~\cite{gb_book}), which
takes as input a set of polynomials $F = \{f_1,\dots, f_s\}$ and
computes its GB $G = \{g_1,\dots,g_t\}$, such that $J = \langle
F\rangle = \langle G\rangle$. Moreover, the polynomials of $F$ and $G$
have the same solution-set, i.e. $V(J) = V(F) = V(G)$. 


%% Buchberger's algorithm can be easily extended to relate a GB of an ideal 
%% with any generating set contained in the ideal. 
%% While computing a GB, we can output 
%% not just the GB $G=\{g_1,\dots,g_t\}$ but also a $t\times
%% s$ matrix $M$ with polynomial entries such that:

%% \vspace{-0.1in}
%% \begin{small}
%% \begin{equation}\label{eqn:matrix}
%% \begin{bmatrix} g_1 \\ g_2 \\ \vdots \\ g_t \end{bmatrix}  =  M \cdot
%% \begin{bmatrix} f_1 \\ f_2 \\ \vdots \\ f_s \end{bmatrix}
%% \end{equation}
%% \end{small}
%% \vspace{-0.1in}

%%:$M$ is a $t\times s$ polynomial matrix.\\

{\it Ideal Membership Test:} The above definition inherently provides
a decision procedure to test for membership of a polynomial in an ideal:
$f$ is a member of an ideal $J$ {\it if and only if} 
% dividing $f$ by the \Grobner basis $G = GB(J)$ gives the remainder 0. 
$f\xrightarrow{GB(J)}_+0$.
When $f\notin J$,
then division by $GB(J)$ results in a non-zero remainder $r$ that is
unique/canonical. 
% \begin{Proposition}[cf. Thm. 1.9.1. (iv) in \cite{gb_book}]
% \label{prop:unique_rem}
% For a \Grobner basis $G$ and a polynomial $f$ in $R$, if
% $f\xrightarrow{G}_+ r_1, f\xrightarrow{G}_+ r_2$, then $r_1 = r_2$. 
% \end{Proposition}


%Therefore, given any ideal $J$, $V_{\Fq}(J) = V_{\Fqbar}(J)
%\cap\Fq^d = V_{\Fqbar}(J) \cap V_{\Fqbar}(J_0) = V_{\Fqbar}(J+J_0) =
%V_{\Fq}(J+J_0)$ (\cite{gao:gf-gb-ms}).

For any element $\varphi\in\Fq$,
$\varphi^q=\varphi$ holds. Therefore, the polynomial $x^q-x$ vanishes
everywhere in $\Fq$, and we call it a {\it vanishing polynomial}. We
denote by $F_0 = \{x_1^q-x_1,\dots,x_d^q-x_d\}$ the set of all vanishing
polynomials in $R$, and  $J_0 = \langle F_0 \rangle$
as the ideal of vanishing polynomials. Further,
$V_{\Fq}(J_0) = \Fq^d$, and for any ideal $J$,
$V_{\Fq}(J)=V_{\Fq}(J+J_0)$. 
% In this work, we represent the $\impl$ $C$ using the ideal $J+J_0$,
% and check for MFR by reasoning about the variety
% $V_{\Fq}(J+J_0)$. 

%% \begin{Definition}
%% Given an ideal $J\subset R$ and $V(J) \subseteq \Fq^d$, the {\it ideal
%% of polynomials that vanish on} $V(J)$ is $I(V(J)) = \{ f \in R :
%% \forall \bm{a} \in V(J), f(\bm{a}) = 0\}$.
%% \end{Definition}
%% %\vspace{-0.08in}

%% If $f$ vanishes on $V_{\Fq}(J)$, then $f \in I(V_{\Fq}(J))$. The Strong
%% Nullstellensatz over finite fields characterizes this ideal:

%% \begin{Theorem}
%% \label{thm:sns}
%%   [The Strong Nullstellensatz over Finite Fields (cf. Thm 2.2 in
%%     \cite{gao:qe-gf-gb})]   
%% Let $J \subset R = \Fq[x_1,\dots,x_d]$ be any ideal, and $J_0=\langle
%% x_1^q-x_1,\dots,x_d^q-x_d\rangle$ be the ideal of vanishing 
%% polynomials in $R$. Then, $I(V_{\Fq}(J)) = J + J_0$.
%% \end{Theorem}

%% \begin{Definition}[Elimination Ideal]
%% Given an ideal $J = \langle f_1, \dots, f_s\rangle \subset
%% \Fq[x_1,\dots,x_d]$, the $l$-th elimination ideal $J_l$ is defined as
%% $J_l = J \cap \Fq[x_{l+1},\dots,x_d]$. 
%% \end{Definition}
%% The ideal $J_l$ is called an elimination ideal because it eliminates
%% variables $x_1,\dots,x_{l}$. Generators of the $l$-th
%% elimination ideal can be computed using \Grobner bases using {\it lex}
%% term orders.
%% \begin{Theorem}[Elimination Theorem]
%% \label{def:elim}
%% Let $J \subset R$ be an ideal and $G$ be its \Grobner basis w.r.t. the
%% lexicographical (lex) order on the variables where $x_1 > x_2 > \cdots
%% > x_d$. Then for every $0 \leq l \leq d$ the set $G_l = G \cap
%% \Fq[x_{l+1},\dots,x_d]$ is a \Grobner basis of the $l^{th}$
%% elimination ideal $J_l$. 
%% \end{Theorem}
% The rectifiability concepts presented in the sequel rely on the above
% results of the Strong Nullstellensatz over finite fields and \Grobner
% basis reduction $f\xrightarrow{GB(J+J_0)}_+ r$.
%using elimination ideals.  
%Our verification and rectification tests follow from the Strong Nullstellensatz
%%%%% RTTO
%% \begin{Definition}
%% \label{def:rtto}
%% \par {\it Reverse Topological Term Order~\cite{lv:tcad2013}:}
%% The computational complexity of Buchberger's algorithm is exponential
%% in the number of variables $n$. As our work is focused on the circuits,
%% we will describe a term order that renders the set of polynomials for 
%% the gates of the circuit, a \Grobner basis itself. This term order 
%% is called Reverse Topological Term Order (RTTO).
%% \par Let $C$ be an arbitrary combinational
%% circuit. Let $\{x_1, \dots$ $, x_d\}$ denote the set of all variables
%% (signals) in $C$. Starting from the primary outputs, perform
%% a {\it reverse topological traversal} of the circuit and order the
%% variables such that $x_k > x_j$ if $x_k$ appears earlier in the
%% reverse topological order. Impose a lex term order $>$ to represent each
%% gate as a polynomial $f_j$, s.t. $f_j = x_k + tail(f_j)$. Then 
%% set of polynomials $\{f_1,\dots,f_s\}$ corresponding to the gates of the circuits 
%% is a \Grobner basis when RTTO is used for ordering.
%% \end{Definition}
%% % \par {\bf Weak Nullstellensatz and Elimination Theory:} 
%% \begin{Theorem}[{\it The Weak Nullstellensatz over finite fields (from
%% Theorem 3.3 in~\cite{gao:gf-gb-ms})}]
%% \label{thm:weak-ns-ff}
%% {\it For a finite field $\Fq$ and the ring $R = \Fq[x_1, \dots, x_d]$, let
%% $J = \langle f_1, \dots, f_s\rangle \subseteq R$, and let $J_0 = \langle
%% x_1^q-x_1, \dots, x_d^q -  x_d\rangle$ be the ideal of vanishing
%% polynomials. Then $V_{\Fq}(J) = \emptyset \iff 1 \in J + J_0 \iff G =
%% GB(J+J_0) = \{1\}$. }
%% \par To find whether a set of polynomials $f_1,\dots,f_s$ have no common
%% zeros in $\Fq$, we can compute the GB $G$ of
%% $\{f_1,\dots,f_s,x_1^q-x_1,\dots,x_d^q-x_d\}$ and see if $G = \{1\}$. 
%% \end{Theorem}
%% We also need to employ notion of difference of varieties in our theoretical
%% section. The equivalent ideal operation is called the quotient of ideals.
%% In terms of varieties, $V_{\Fq}(J_1:J_2) = V_{\Fq}(J_1) \setminus V_{\Fq}(J_2)$.
%% \par The computer algebra tools like SINGULAR~\cite{DGPS_410} contain implementations for 
%% computing elimination ideals and quotient of ideals.
%% \subsection{Circuit Polynomials}
%% \label{composite_field}
%% For a given data-path size $n$, $q=2^n$ is
%% chosen to
% takes W which takes values in $\Fkm$.
% change f to fspec.
% f1 to fs are gates of the circuit. 
