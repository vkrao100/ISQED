\begin{small}
\begin{Theorem}\label{thm:Pk}
Given primitive polynomials $P_m(X), P_n(X)$ of the fields $\F_{2^m}, \F_{2^n}$, respectively, 
with $P_m(\alpha^\mu)=P_n(\alpha^\lambda)= 0$ (Eqn.~(\ref{word-level-rep})), 
obtain UPFs $P_n(X^\lambda) = P_{n_1}^{a_1}(X)\cdots P_{n_l}^{a_l}(X)$ and similarly
$P_{m}(X^\mu)=P_{m_1}^{b_1}(X)\cdots P_{m_g}^{b_g}(X)$ in $\F_2[X]$. 
Then there exists a $P_{n_i}(X) \in \{P_{n_1}(X),\dots,P_{n_l}(X)\}$ 
and a $P_{m_j} \in \{P_{m_1}(X),\dots,P_{m_g}(X)\}$, such that
\begin{enumerate}
\item $P_k(X) = P_{n_i}(X) = P_{m_j}(X)$, and
\item $P_k(X)$ is a degree-$k$ primitive polynomial in $\F_{2}[X]$
  such that $P_{k}(\alpha)=0$. 
\end{enumerate}
\end{Theorem}
\end{small}

\begin{proof}

{\it 1:} Let $P(X)$ be the minimal polynomial of $\alpha$ in $\F_2$. Since $\alpha$ is a primitive element of the field $\Fkk$, 
we have $\Fkk = \F_2(\alpha)$, and hence $P(X)$ is also primitive, i.e. $P(\alpha)=0$. But, we have $P_m(\alpha^\mu)= 0$,  
which implies $\alpha$ is a root of $P_m(X^\mu)$ as well. Hence, $P(X) | P_m(X^\mu)$, and likewise, $P(X) | P_n(X^\lambda)$.
So, $P(X)$ is a common factor of UPFs obtained from $P_{m}(X^\mu)$ and $P_{n}(X^\lambda)$. Further, $P(X)$ is a primitive polynomial to begin with.

{\it 2:} Let $P(X)$ be a common primitive factor obtained from UPFs of $P_{m}(X^\mu)$ and $P_{n}(X^\lambda)$, and $\alpha$ be a root of it, i.e. 
$\F_2(\alpha)$ is a $k$-bit extension of the field $\F_2$, denoted as $\F_2(\alpha) / \F_2:k$. Since $P(X)$ is primitive, $\alpha$ is primitive.
Since, $P_m(\alpha^\mu)=0$ and $P_m(X)$ is primitive, which implies $\F_2(\alpha^\mu) / \F_2:m$, similarly, $\F_2(\alpha^\lambda) / \F_2:n$.  
\end{proof}
