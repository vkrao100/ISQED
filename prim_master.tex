\documentclass[10pt,twocolumn]{IEEEtran}
%\documentclass{article}

%% set page size to US letter
\special{papersize=8.5in,11in}
\setlength{\pdfpageheight}{\paperheight}
\setlength{\pdfpagewidth}{\paperwidth}

\DeclareMathAlphabet{\mathtsl}{OT1}{ptm}{m}{sl}
\RequirePackage{amssymb}
\usepackage{hyperref}
\usepackage{xspace}
\usepackage{algorithm}
\usepackage[noend]{algpseudocode}
\usepackage{amsbsy}
\usepackage{amsthm}
\usepackage{graphicx}
\usepackage{helvet}
\usepackage{enumerate}
\usepackage{amsmath}
\usepackage{amstext}
\usepackage{amsfonts}
\usepackage{graphicx}
\usepackage{multirow}
\usepackage{subfig}
\usepackage{comment}
\usepackage{cases}
\usepackage{xcolor}
\usepackage{epstopdf}
\usepackage[normalem]{ulem}
\usepackage{diagbox}
\usepackage{cleveref}
\usepackage{bm}
\usepackage{setspace}

% \usepackage{titlesec}
% \titlespacing*{\section}{0pt}{1.1\baselineskip}{\baselineskip}

\newcommand{\B}{{\mathbb{B}}}
\newcommand{\Z}{{\mathbb{Z}}}
\newcommand{\R}{{\mathbb{R}}}
\newcommand{\Q}{{\mathbb{Q}}}
\newcommand{\N}{{\mathbb{N}}}
% \newcommand{\C}{{\mathbb{C}}}
\newcommand{\beqarr}{\begin{eqnarray}}
\newcommand{\eeqarr}{\end{eqnarray}}
\newcommand{\ov}{\bar}
\newcommand{\xor}{\bigoplus}
\newcommand{\Fm}{{\mathbb{F}}}
\newcommand{\myfontsize}{\fontsize{7}{9}\selectfont}
\newcommand{\Ftwo}{{\mathbb{F}}_{2}}
\newcommand{\Ftwob}{\overline{\mathbb{F}}_{2}}
\newcommand{\Zn}{{\mathbb{Z}}_{n}}
\newcommand{\Zp}{{\mathbb{Z}}_{p}}
\newcommand{\F}{{\mathbb{F}}}
\newcommand{\FF}{{\mathcal{F}}}
\newcommand{\Fbar}{{\overline{\mathbb{F}}}}
\newcommand{\Fq}{{\mathbb{F}}_{q}}
\newcommand{\Fqbar}{\overline{\mathbb{F}}_q}
\newcommand{\Fkk}{{\mathbb{F}}_{2^k}}
\newcommand{\Fn}{{\mathbb{F}}_{2^n}}
\newcommand{\Zkk}{{\mathbb{Z}}_{2^k}}
\newcommand{\Fkkx}[1][x]{\ensuremath{\mathbb{F}}_{2^k}[#1]\xspace}
\newcommand{\Grobner}{Gr\"{o}bner }
\newcommand{\bi}{\begin{itemize}}
\newcommand{\ei}{\end{itemize}}
\newcommand{\impl}{{\it Impl}}
\newcommand{\spec}{{\it Spec}}
% \newcommand{\spec}{{\it Spec}\xspace}
% \newcommand{\impl}{{\it Impl}\xspace}
\newcommand{\idealf}{{F = \{f_1, \dots, f_s\}}}
\newcommand{\idealj}{{J = \langle f_1, \dots, f_s\rangle}}
\newcommand{\idealg}{{J = \langle g_1, \dots, g_t\rangle}}
\newcommand{\vfqj}{{V_{\Fq}(J)}}
\newcommand{\vfkkj}{{V_{\Fkk}(J)}}
% \newcommand{\G}{{\mathcal{G}}}
% \newcommand{\alert}[1]{\textcolor{red}{#1}}
\newcommand{\Fkn}{{\mathbb{F}}_{2^n}}
\newcommand{\Fkm}{{\mathbb{F}}_{2^m}}
\newcommand{\vfqjo}{{V_{\Fq}(J_0)}}
\newcommand{\vfbqj}{{V_{\overline{\Fq}}(J)}}
\newcommand{\vfbqjo}{{V_{\overline{\Fq}}(J_0)}}
\newcommand{\vfbqjjo}{{V_{\overline{\Fq}}(J+J_0)}}
\newcommand{\In}{\mathcal{I}_n}
\newcommand{\M}{\mathcal{M}}
\newcommand{\Ic}{\mathcal{I}_c}
\newcommand{\Oa}{\mathcal{O}_a}
\newcommand{\Oao}{\mathcal{O}_{a_1}}
\newcommand{\Oat}{\mathcal{O}_{a_2}}

%%% Added by Utkarsh %%%
\newcommand{\Va}{{V_A}}
\newcommand{\Vb}{{V_B}}
\newcommand{\Vc}{{V_C}}
\newcommand{\Vbc}{{V_{B,C}}}
\newcommand{\Vabc}{{V_{A,B,C}}}
\newcommand{\Vac}{{V_{A,C}}}
\newcommand{\acf}{\bar{F}_q}
\newcommand{\Vacf}{V_{\bar{F}_q}}
\newcommand{\w}{\wedge}
\newcommand{\al}{\alpha}
\newcommand{\ga}{\gamma}
\newcommand{\be}{\beta}
\newcommand{\vpi}{V_{X_{PI}}}
\newcommand{\uc}{U(X_{PI})}
\newcommand{\xpi}{X_{PI}}
\newcommand{\fqring}{\Fq[x_1,\dots,x_d]}
\newcommand{\ftring}{\F_2[x_1,\dots,x_d]}
\newcommand{\ftkring}{\Fkk[x_1,\dots,x_d]}
\newcommand{\ftkwring}{\Fkk[x_1,\dots,x_d,Z,A,B,W]}
\newcommand{\debug}[1]{\textcolor{gray}{[ #1 ]}}
\newcommand{\blu}{\color{blue}}
\newcommand{\green}{\color{green}}
\newcommand{\yellow}{\color{yellow}}
\newcommand{\red}{\color{red}}
\newcommand{\Cct}{!{\vrule width 1.5pt}}
\newcommand{\Rrt}{\noalign{\hrule height 1.5pt}}
\newcommand{\mb}[1]{$\mathbf{#1}$}
\newcommand{\td}{$\textsuperscript{\textdagger}$}
% \newcommand{\bif[1]}{\bf{\it #1}}

%%%%%%%%%%%%%%%%%%%%%%%%

% \algnewcommand\algorithmicinput{\textbf{Assume:}}	
% \algnewcommand\Assume{\item[\algorithmicinput]}
% \algdef{SE}[DOWHILE]{Do}{doWhile}{\algorithmicdo}[1]{\algorithmicwhile\ #1}%

% \algnewcommand\algorithmicforeach{\textbf{for each}}
% \algdef{S}[FOR]{ForEach}[1]{\algorithmicforeach\ #1\ \algorithmicdo}

% New command for the line spacing.
% \newcommand{\ls}[1]
%     {\dimen0=\fontdimen6\the\font
%      \lineskip=#1\dimen0
%      \advance\lineskip.5\fontdimen5\the\font
%      \advance\lineskip-\dimen0
%      \lineskiplimit=.9\lineskip
%      \baselineskip=\lineskip
%      \advance\baselineskip\dimen0
%      \normallineskip\lineskip
%      \normallineskiplimit\lineskiplimit
%      \normalbaselineskip\baselineskip
%      \ignorespaces
%     }
% New command for the table bnotes.
\def\tabnote#1{{\small{#1}}}

% \theoremstyle{definition}

%the following is for space before and after align or other equation environment.
\newtheorem{Algorithm}{Algorithm}[section]
\newtheorem{Definition}{Definition}[section]
\newtheorem{Example}{Example}[section]
\newtheorem{Proposition}{Proposition}[section]
\newtheorem{Lemma}{Lemma}[section]
\newtheorem{Theorem}{Theorem}[section]
\newtheorem{Proof}{Proof}[section]
\newtheorem{Corollary}{Corollary}[section]
\newtheorem{Conjecture}{Conjecture}[section]
\newtheorem{Problem}{Problem}[section]
\newtheorem{Notation}{Notation}[section]
\newtheorem{Setup}{Problem Setup}[section]

%to autoref throermes and definitions
\providecommand*\Theoremautorefname{Theorem}
\providecommand*\Lemmaautorefname{Lemma}
\providecommand*\Definitionautorefname{Definition}

%%set spacing between table columns
\setlength{\tabcolsep}{3pt}
% \setlength\intextsep{0pt}
\setcounter{secnumdepth}{3}

\begin{document}
%% \setlength{\abovedisplayskip}{0pt}
%% \setlength{\belowdisplayskip}{0pt}
%% \setlength{\abovedisplayshortskip}{0pt}
%% \setlength{\belowdisplayshortskip}{0pt}
\title{\Large{ Word-Level Multi-Fix Rectification of Finite Field Arithmetic Circuits} }
  % \thanks{This research is funded in part by the
  %  US National Science Foundation grants CCF-1619370 and
  %  CCF-1320385.}}

% \author{Vikas Rao$^1$, Utkarsh Gupta$^1$, Priyank Kalla$^1$, Irina Ilioaea$^2$, and Florian Enescu$^2$\\
% $^1$Electrical \& Computer Engineering, University of Utah\\
% $^2$Mathematics \& Statistics, Georgia State University \vspace{-0.2in}
% }

% \institute{}
%\thispagestyle{empty}

%\maketitle
%%%%%%%%%%%%%%%%%%%% Include your files here %%%%%%%%%%%%%%%%%%%%%
% \input{abstract_vk_pk.tex}
% \input{intro_vk_pk.tex}
% \input{prelim.tex}
% \input{verify.tex}
% \input{comp_setup.tex}
% \input{rectify.tex}
% % \input{unknown.tex}
% \input{experiments.tex}
\begin{small}
\begin{Theorem}\label{thm:Pk}
Given primitive polynomials $P_m(X), P_n(X)$ of the fields $\F_{2^m}, \F_{2^n}$, respectively, 
with $P_m(\alpha^\mu)=P_n(\alpha^\lambda)= 0$ (Eqn.~(\ref{word-level-rep})), 
obtain UPFs $P_n(X^\lambda) = P_{n_1}^{a_1}(X)\cdots P_{n_l}^{a_l}(X)$ and similarly
$P_{m}(X^\mu)=P_{m_1}^{b_1}(X)\cdots P_{m_g}^{b_g}(X)$ in $\F_2[X]$. 
Then there exists a $P_{n_i}(X) \in \{P_{n_1}(X),\dots,P_{n_l}(X)\}$ 
and a $P_{m_j} \in \{P_{m_1}(X),\dots,P_{m_g}(X)\}$, such that
\begin{enumerate}
\item $P_k(X) = P_{n_i}(X) = P_{m_j}(X)$, and
\item $P_k(X)$ is a degree-$k$ primitive polynomial in $\F_{2}[X]$
  such that $P_{k}(\alpha)=0$. 
\end{enumerate}
\end{Theorem}
\end{small}

\begin{proof}

{\it 1:} Let $P(X)$ be the minimal polynomial of $\alpha$ in $\F_2$. Since $\alpha$ is a primitive element of the field $\Fkk$, 
we have $\Fkk = \F_2(\alpha)$, and hence $P(X)$ is also primitive, i.e. $P(\alpha)=0$. But, we have $P_m(\alpha^\mu)= 0$,  
which implies $\alpha$ is a root of $P_m(X^\mu)$ as well. Hence, $P(X) | P_m(X^\mu)$, and likewise, $P(X) | P_n(X^\lambda)$.
So, $P(X)$ is a common factor of UPFs obtained from $P_{m}(X^\mu)$ and $P_{n}(X^\lambda)$. Further, $P(X)$ is a primitive polynomial to begin with.

{\it 2:} Let $P(X)$ be a common primitive factor obtained from UPFs of $P_{m}(X^\mu)$ and $P_{n}(X^\lambda)$, and $\alpha$ be a root of it, i.e. 
$\F_2(\alpha)$ is a $k$-bit extension of the field $\F_2$, denoted as $\F_2(\alpha) / \F_2:k$. Since $P(X)$ is primitive, $\alpha$ is primitive.
Since, $P_m(\alpha^\mu)=0$ and $P_m(X)$ is primitive, which implies $\F_2(\alpha^\mu) / \F_2:m$, similarly, $\F_2(\alpha^\lambda) / \F_2:n$.  
\end{proof}

%%%%%%%%%%%%%%%%%%%% The bibliography %%%%%%%%%%%%%%%%%%%%%%%%%%%%

%List of our publications
% in utkarsh.bib
% Utkarsh:ETS19
% Utkarsh:book-chapter
% Utkarsh:VLSI18
% Utkarsh:IWLS18
% Utkarsh:tcad17
% Arpitha:MSthesis-2019
% % in vikas.bib
% Vkrao:IWLS18
% Vkrao:FMCAD18
% signal selection papers
% SS_Roland:DAC19
% SS_Fujita:ISCAS19
% SS_Fujita:ISQED17
% SS_Alan:DAC18
% SS_Huang:ICCAD17
% SS_Roland:DAC18
% SS_Goersch:ASPDAC17
% %+multi-fix
% MF_Roland:ICCAD10
% MF_Huang:DAC11
% MF_Huang:DATE12
% MF_Rolf:ISVLSI18
% %+synthesis
% SYN_Huang:ICCAD19
% 

% \bibliographystyle{IEEEtran}
% \bibliography{vikas,utkarsh,tim,xiaojun,logic}

\end{document}

%%%%%%%%%%%%%%%%%%%%%%%%%%%  End of IEEEsample.tex  %%%%%%%%%%%%%%%%%%%%%%%%%%%
