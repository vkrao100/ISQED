\section{A Unified Framework for MFR}\label{sec:comps}

%After detecting the presence of bug(s) in the design,
% Having been provided with a set of $m$ targets from $C$, our objective now is to 
% ascertain whether these internal nets admit MFR; i.e. does there exist an $m$-bit patch
% function which rectifies $C$ at these targets? 
% We assume that these targets were provided beforehand or the heuristics 
% proposed in~\cite{SS_Alan:DAC18,SS_Fujita:ISCAS19,SS_Roland:DAC19} were applied to select the same.
%In order to ascertain whether these $m$-targets
%indeed admit MFR, we set up a unified algebraic decision framework as
%follows. 

%\subsection{Word-level Formulation of Patch
%Function}\label{subsec:word_form}
{\it Word-level Rectification Patch:} 
Given a set of $m$-targets, let $W = \{w_0,w_1,\dots,w_{m-1}\}$ denote the $m$-bit vector word,
over which MFR check is to be performed. 
%where the admissibility of the $m$-bit rectification patch is to be
%ascertained.
Here, $w_0,\dots,w_{m-1}\in \{x_1,\dots,x_d\}$
are internal nets of $C$. A rectification patch corresponds to 
a function mapping $f_W: \F_2^{|X_{PI}|}\rightarrow\F_{2^m}$, and
represented as a polynomial function $U(X_{PI}) \in
\F_{2^m}[X_{PI}]$. Here, $U$ is an {\it unknown} polynomial
with variables from $X_{PI}$, and coefficients from $\F_{2^m}$. MFR
then implies that, algebraically, $W = U(X_{PI})$ rectifies the
circuit. 

Such a setup {\it interprets} $U$ as a polynomial function which
evaluates to values in the field $\F_{2^m}$. To construct $\F_{2^m}$,
we {\it select} a degree-$m$ primitive polynomial $P_m(X)\in\F_2[X]$, with
$\beta$ as a root ($P_m(\beta)=0$), such that 
$\F_{2^m} = \F_2[X] \pmod{ P_m(X)}$. Note that any degree-$m$
primitive polynomial ($P_m(X)$) can be selected. This implies that
$\beta$ is a PE of $\F_{2^m}$. Thus, the rectification patch polynomial
function can be represented as: $f_W: W + U(X_{PI}) = W +
\sum_{i=0}^{m-1}\be^iw_i$.  
% \begin{Example}
% \label{target_ex}
% Continuing with our example Ex. \ref{verify_ex}, we analyze the remainder $r$ of verification.
% Note that since the intersection of the fan-in cones of all outputs in $\Oa$ is empty, 
% $C$ doesn't admit single-fix rectification. Further, let $\Ic$ represent the initial cost
% associated with each net in $\In$ assigned based on their respective topological levels from PI.
% Since, each of the nets $\{d_0,e_0,d_2,d_5\}$ lie in the intersection of 2 outputs, we change their 
% cost accordingly.

% {\small
% \begin{flalign*}
% &coefficients(r) = \{ \ga^0,\ga^1,\ga^2\} \implies \Oa = \{z_0,z_1,z_2\} \\
% &\In = \{z_0,z_1,z_2,d_0,f_0,e_2,e_3,e_0,e_1,rr_0,d_6,d_7,d_8,d_1,d_2,d_3,r_0,\\
% &\quad d_5,rr_1,r_1,rr_2,r_2,r_3,rr_3,d_4,r_4,rr_4\} \\
% &\Ic = \{7,7,7,6,6,6,6,5,3,3,5,5,5,4,4,2,2,2,2,1,1,0,0,0,\\
% &\quad -1,-1,-1\} 
% \end{flalign*}
% }
% We partition the set $\Oa$ into 2 ($m$) distinct non-empty subsets
% and compute the corresponding intersection covers. Any individual
% combination of nets from these subsets will always cover the set $\Oa$. 
% {\small
% \begin{flalign*}
% &\Oa = \{\{z_0,z_1\},\{z_2\}\} \\
% &\M = \{\M_0:\{r_4,d_4,r_3,r_2,r_1,r_0,d_3,e_1\},\\
% &\quad\M_1:\{rr_4,rr_3,rr_2,rr_1,d_5,rr_0,d_6,d_7,d_8,e_2,e_3,z_2\}\}
% \end{flalign*}
% }

% We will pick net $r_3$ from $\M_0$ as the first target ($w_0=r_3$) and $rr_3$ 
% from $\M_1$ as the second target ($w_1=rr_3$). The multi-fix patch
% will be modeled as a $2$-bit operand over ${\mathbb{F}}_{2^2}$ constructed as 
% $\Ftwo[X]\pmod{P_2(X)}$ with $\be$ as one of its root, i.e. $P_2(\be)=0$.
% Fig. \ref{fig:mas_bug_W}(b) illustrates $W$ representing the word-level 
% patch as a function of nets $r_3$ and $rr_3$,$ f_w: W+r_3+\be \cdot rr_3$. 
% RTTO with modified variable order:
% $\{Z\}>\{A>B\}>\{z_0>z_1>z_2\}>\{d_0>f_0>e_2>e_3\}>\{e_0>e_1>rr_0>d_6>d_7>d_8\}>\{d_1>d_2>d_3>r_0>d_5>rr_1\}>\{r_1>rr_2\}>\{r_2>{\bf{W}}>r_3>rr_3\}>\{d_4>r_4>rr_4\}>\{a_0>a_1>a_2>b_0>b_1>b_2\}.$ 
% \end{Example}

%\subsection{Composite Field Framework}\label{subsec:comp_fwrk}
{\it Composite Field Framework:} The circuit $C$ is modeled over
$\Fkn$ and the rectification patch is modeled over $\Fkm$. However, the \Grobner basis 
computations for MFR can only be performed over a ring with
coefficients from a single field. We select the field $\Fkk$
of $2^k$ elements, where $k$ is the least common multiple of $m$ and
$n$ ($k=LCM(m,n)$). Thus $\Fkk$ corresponds to the smallest field
containing both $\F_{2^m}$ and $\F_{2^n}$.
% , i.e. $\Fkn \subset \Fkk$ and $\Fkm \subset \Fkk$.
% , thereby allowing every $m$-bit and $n$-bit vector word
% to be construed as elements in a single field $\Fkk$.

\par To construct $\Fkk$, we must now identify
a degree-$k$ primitive polynomial $P_{k}(X)\in\F_2[X]$, and a
corresponding primitive element $\alpha$ such that
$P_k(\alpha)=0$. While there may exist many degree-$k$ primitive
polynomials, the selected $P_k(X)$ needs to conform to specific
conditions imposed by the {\it given} $P_n(X)$, and
the {\it selected} $P_m(X)$.

%, such that $P_n(\gamma) =
%P_m(\beta)=0$.
%%%For this purpose, we exploit the concepts of polynomial
%%%factorization in finite fields \cite{factorization_fq:survey}. 

%Further, care
%should be taken while  
%composing the vanishing polynomials: $x_i^2-x_i$ for the bit-level
%variables, $W^{2^m}-W$ for the $m$-bit bit-vectors, and 
%$Z^{2^n}-Z$ for the $n$-bit bit-vectors.

%% To select the primitive polynomial $P_k(X)$, and a corresponding
%% primitive element $\alpha$, we employ the concept of univariate
%% polynomial facorization in finite fields~\cite{factorization_fq:survey}.

%% \begin{Problem}{[Univariate Polynomial Factorization (UPF)]}
%% Given a monic univariate polynomial $f \in \Fq[x]$, where $\Fq$ is
%% any finite field, find a complete factorization $f = f_1^{e_1}\cdot$ 
%% $\dots f_l^{e_l}$, where $f_1,\dots, f_l$ are pairwise
%% distinct monic irreducible polynomials in $\Fq[x]$, and
%% $e_i\in \Z_{\geq 1}$ are positive integers.
%% \end{Problem}

%% Efficient algorithms to obtain such a factorization in finite fields
%% are well-known \cite{factorization_fq:survey}. %\cite{kaltofen:issac97}, 
%and they are implemented in computer algebra and computational number
%theory tools, such as in \cite{DGPS_410} and \cite{ntl}. We employ
% and we employ the implementation of \cite{ntl} to perform the
% aforementioned UPF. 
Given the primitive elements $\beta, \gamma$ of the fields 
$\F_{2^m},\F_{2^n}$, respectively, let $\alpha$ be a PE of $\Fkk$,
such that $\Fkk \supseteq \F_{2^m},\Fkn$. For the desired unified
computational framework, we represent $\beta, \gamma$ in terms
of $\alpha$. As $\alpha \in \Fkk, \alpha^{2^k-1}=1$. Similarly,
$\beta\in \Fkm$ implies that $\beta^{2^{m}-1}=1$, and $\gamma\in\Fkn$
makes $\gamma^{2^n-1}=1$. Equating
$\be^{2^m-1}=\ga^{2^n-1}=\alpha^{2^k-1} = 1$, we obtain:
% This can be achieved as:
\begin{equation}\label{word-level-rep}
\begin{split}
 \be &= \al^{(2^k-1)/(2^m-1)} = \al^{\mu}\\
 \ga &= \al^{(2^k-1)/(2^n-1)} = \al^{\lambda}
 \end{split}
\end{equation}

With this setup, we show that the desired primitive polynomial
$P_k(X)$ for $\F_2(\alpha)$ exists as a degree-$k$ {\bf common factor}
of $P_n(X^{\lambda})$ and $P_m(X^{\mu})$, and it can be found by
performing univariate polynomial factorizations (UPFs) of
$P_n(X^{\lambda})$ and $P_m(X^{\mu})$ in the base field
$\F_2[X]$. Efficient algorithms to perform UPFs in finite fields are
well-known \cite{factorization_fq:survey}, which can be employed for
this purpose. 

%\begin{small}
\begin{Theorem}\label{thm:Pk}
Given primitive polynomials $P_n(X), P_m(X)$ $\in \F_2[X]$ of degrees
$n,m$, respectively, along with $P_n(\gamma)= P_m(\beta)=0$. Let 
$k=LCM(m,n)$ and $\alpha$ be a PE of $\Fkk$, such that
$\beta=\alpha^\mu$ and $\gamma=\alpha^\lambda$, where
$\mu={(2^k-1)/(2^m-1)}$ and $\lambda={(2^k-1)/(2^n-1)}$.
%Obtain UPFs $P_n(X^\lambda)=P_{n_1}^{a_1}(X)\cdot
%P_{n_2}^{a_2}(X)\cdots P_{n_l}^{a_l}(X)$, and
%$P_{m}(X^\mu)=P_{m_1}^{b_1}(X)\cdots P_{m_g}^{b_g}(X)$, in
%$\F_2[X]$. Then there exists a $P_{n_i}(X) \in
%\{P_{n_1}(X),\dots,P_{n_l}(X)\}$ and a $P_{m_j} (X) \in
%\{P_{m_1}(X),\dots,P_{m_g}(X)\}$ such that $P_k(X) = P_{n_i}(X) =
%P_{m_j}(X)$; i.e.
Perform UPF of $P_n(X^{\lambda})$ and $P_m(X^{\mu})$ in $\F_2[X]$. 
Then there exists a polynomial $P_k(X) \in \F_2[X]$ as a common factor
of $P_m(X^\mu)$ and $P_n(X^\lambda)$, such that $P_k(X)$ is a 
degree-$k$ primitive polynomial with $P_{k}(\alpha)=0$.
\end{Theorem}
%\end{small}

\begin{proof}
Since %$\F_2 \subset \Fkn, \Fkm \subset \Fkk$, and
$\alpha$ is given as a PE of $\Fkk$,  the minimal polynomial of
$\alpha$ is a primitive polynomial. Denote it as $P_k(X)$. Then
$P_k(\alpha)=0$, i.e. $\alpha$ is a root of $P_k(X)$. However,
$P_m(\beta)=0 \implies P_m(\alpha^\mu)=0$. So $P_m(X^\mu)$ also has
$\alpha$ as a root. This implies that $P_k(X)$ divides $P_m(X^\mu)$:
$P_k(X) ~| ~ P_m(X^\mu)$. Similarly, $P_k(X) ~|~ P_n(X^\lambda)$. So
$P_k(X)$ is a common factor of both $P_m(X^\mu)$ and
$P_n(X^\lambda)$. Since $P_m(X^{\mu}), P_n(X^{\lambda}) \in \F_2[X]$,
and their UPFs are also performed in $\F_2[X]$, $P_k(X) \in
\F_2[X]$. Moreover, as $\alpha$ is a PE of $\Fkk$, and $P_k(X)$ is the
minimal polynomial of $\alpha$, we have that $P_k(X)$ is a primitive
polynomial as $\Fkk = \F_2(\alpha)$. Furthermore, by definition, as
the minimal polynomial of $\alpha$, $P_k(X)$ has degree
$=k$. This completes the proof. 
\end{proof}
% The polynomial $P_k(X)$ obtained from Thm. \ref{thm:Pk} can be used to
% construct field $\Fkk = \F_2[X] \pmod{ P_k(X)}$, which encompasses the
% fields corresponding to the circuit ($\Fkn$) and the patch
% ($\Fkm$). Moreover, their respective PE's $\beta, \gamma$, can also be
% related to the PE $\alpha$ of $\Fkk$ by means of
% Eqn. (\ref{word-level-rep}). 


\begin{Example}
\label{composite_ex}
{\it 
Continuing with our rectification example of Fig. \ref{fig:mas_bug_W},
assume that we are given $m=2$ targets: $w_0=r_3$, and $w_1=rr_3$.
The multi-fix patch is modeled as a $2$-bit
operand over $\F_{2^2}=\Ftwo[X]\pmod{P_2(X)=X^2+X+1}$ with
$P_2(\be)=0$. It is given that the 3-bit circuit ($n=3$) is designed
over {\small $\F_{2^3}=\F_2[X]\pmod{P_3(X)=X^3+X+1}$}.  
%Note that we selected $P_2(X)=X^2+X+1$ since that is the only
%primitive polynomial of degree 2. 
With $k=LCM(m,n)=6$, $P_6(X)$ is computed as follows:

Using Eqn. (\ref{word-level-rep}), we have $\gamma=\alpha^9,\beta=\alpha^{21}$. Perform UPFs:
% \begin{multline*}
% 	P_3(X^9) = (X^9)^3+(X^9)+1 =
%   (X^6+X^5+1)(X^6+X^5+X^2+X+1)\\(X^6+X^4+X^3+X+1)(X^6+X^4+X^2+X+1)(X^3+X+1);\\
% 	P_2(X^{21}) = (X^{21})^2+(X^{21})+1 =
%   (X^6+X^5+1)(X^6+X^5+X^2+X+1)\\(X^6+X^4+X^3+X+1)(X^6+X^5+X^3+X^2+1)\\
%   (X^6+X^5+X^4+X+1)(X^6+X+1)(X^6+X^3+1);
% \end{multline*}
\bi
\item ${\scriptstyle P_3(X^9) = (X^9)^3+(X^9)+1 =
  (X^6+X^5+X^2+X+1)(X^6+X^5+1)}$\\${\scriptstyle(X^6+X^4+X^3+X+1)(X^6+X^4+X^2+X+1)(X^3+X+1);}$
\item ${\scriptstyle P_2(X^{21}) = (X^{21})^2+(X^{21})+1 =
  (X^6+X^5+X^2+X+1)}$\\${\scriptstyle (X^6+X^5+1)(X^6+X^4+X^3+X+1)(X^6+X^5+X^3+X^2+1)
  }$\\${\scriptstyle(X^6+X^5+X^4+X+1)(X^6+X+1)(X^6+X^3+1);}$
\ei

Among the degree-6 common factors of ${\scriptstyle P_3(X^9)}$ and
${\scriptstyle P_2(X^{21})}$,  ${\scriptstyle X^6+X^5+X^2+X+1}$ is omitted
since it is non-primitive.  We choose ${\scriptstyle P_6(X) =
  X^6+X^5+1}$ as the required $P_k(X)$. 

Note that if we (incorrectly) choose ${\scriptstyle P_6(X)=X^6 + X^3 +1}$, then it
implies that for its root $\alpha$, we have:
\begin{align*}
\alpha^6 + \alpha^3 + 1 &= 0\\
 (\alpha^3)(\alpha^6 +\alpha^3 + 1) &= 0 ~(\text{multiply by}~\alpha^3)\\
\alpha^9 + \alpha^6 + \alpha^3 &= 0 \\
 \gamma + 1 &= 0
\stepcounter{equation}\tag{\theequation}\label{ll} 
\end{align*}

as $\alpha^9 = \gamma$ and $(\alpha^6 + \alpha^3) = 1$. However,
$\gamma \neq 1$, as $\gamma$ is a PE of $\Fkn$. Therefore, selecting
$P_k(X)$ that does not conform to Thm. \ref{thm:Pk} would lead to
erroneous results.
}
\end{Example}
