\section{Introduction}\label{sec:intro}
Formal verification checks whether a circuit implementation
($\impl$) conforms to its specification ($\spec$).  
In cases where verification detects the presence of bugs,
debugging and rectification are performed. Rectification entails
identifying candidate nets (targets) %followed by a rectifiability check, 
and determining whether the circuit can be patched at these 
targets. If the targets admit rectification,
corresponding rectification functions are computed which make 
the $\impl$ conform to its $\spec$.
This problem manifests itself in engineering change orders (ECO) 
-- where a current $\impl$ needs to be
rectified (preferably, with local modifications) to match the ECO-modified
$\spec$. As a result, the problem
% Since early formulations of this problem were introduced \cite{Madre:ICCAD89,Sadowska:DAC95}, 
has witnessed renewed interest by the logic synthesis, testing and verification
communities~\cite{MF_Huang:DATE12,SS_Fujita:ISCAS19,SS_Roland:DAC19},
etc. These approaches are successful in rectifying random-logic
circuits which are employed in control-dominated applications; 
However, they
are infeasible for the rectification of arithmetic circuits.

\par
{\it This paper addresses the problem of ascertaining the rectifiability 
of buggy finite field arithmetic circuits at a given set of $m$ targets
against a polynomial function (\spec)~over finite fields}.
Such circuits find application in cryptography and error-control
codes. As arithmetic bugs may lead to security vulnerabilities
\cite{crypto:bug_attacks}, their rectification is of utmost
importance.
Our approach models the set of $m$ targets as a bit-vector, enabling 
word-level reasoning, and utilizes techniques 
from Symbolic Computer Algebra (SCA) to determine rectifiability. 
The subsequent problem of computing rectification functions for these
$m$ targets is beyond the scope of this paper. {\it Word-level
  rectifiability checking} for arithmetic circuits is a challenging
problem in its own right, and this manuscript covers its various
facets. 

\par 
{\it Prior Work:}  
Recent works attempt rectification using SCA techniques for finite field 
circuits~\cite{Utkarsh:VLSI18,Vkrao:FMCAD18}, and for integer
arithmetic circuits~\cite{farimah:2017:1,MF_Rolf:ISVLSI18}. 
However, these algebraic approaches address only
{\it single-fix} rectification -- where rectification is attempted
only at a {\it single net}. This is too restrictive, and depending on
the nature of the bugs, the circuit may not admit single-fix
rectification. In such cases, the correction has to be
attempted at multiple targets. This is
called {\it multi-fix} rectification in literature. The focus of
this paper is $m$-target {\it multi-fix rectifiability} (MFR) of finite field circuits. 

\par 
Contemporary approaches formulate the rectifiability 
checking problem inherently as part of a rectification procedure. 
They use quantified Boolean formula solving,
Craig Interpolation, or iterative SAT
solving~\cite{MF_Huang:DATE12,SS_Fujita:ISCAS19,SS_Alan:DAC18}.  
%  MF_Roland:ICCAD10,
% The preliminary MFR check technique~\cite{MF_Huang:DAC11} is incomplete
% in deriving the necessary and sufficient conditions for an MFR check when a set of 
% targets are provided a priori. The authors later present a complete approach \cite{MF_Huang:DATE12} 
The approach~\cite{MF_Huang:DATE12} considers multiple targets simultaneously and 
formulates the MFR check as a QBF. 
The QBF is then translated to a SAT problem and solved iteratively using a cofactor reduction technique.
More recent techniques use this QBF formulation coupled with heuristics that improve
resource awareness in patch generation \cite{SS_Alan:DAC18}, and the selection of effective targets \cite{SS_Fujita:ISCAS19}. 
The recent symbolic sampling approach~\cite{SS_Roland:DAC19} uses simulation to 
identify targets and ascertain rectifiability.
However, models based on Boolean functions and SAT solvers are infeasible to perform rectifiability checking 
on arithmetic circuits. Our experiments show that contemporary SAT solvers fail to rectify finite field circuits beyond 16-bit operands.

% Recently, an explicit enumeration approach to identify targets
% was presented in~\cite{SS_Roland:DAC19}. The authors 
% enumerate prime cubes of the characteristic function of all feasible targets 
% and use them to construct an explicit list of targets. 
% To achieve scalability, the method proposes 
% modeling and analyzing Boolean reasoning queries in symbolic sampling domain. 
% However, the authors do not discuss the application of the proposed technique to 
% arithmetic circuits, hence an efficient/scalable solution is still desired.
% The signal selection heuristic (main contribution) proposed in \cite{SS_Roland:DAC19}
% is orthogonal to the setup and approach presented in our paper.
% However, the authors don't discuss the application of the proposed technique to 
% arithmetic circuits, hence an efficient/scalable solution is still desired. 

% In contrast, our proposed approach being word-level, ascertains
% $m$-target rectifiability in one decision procedure.

%or resolve a combination of such objectives, such as the symbolic
%sampling approach of 
% \cite{SS_Roland:DAC19}. {\it Limitations:} Since
% the patches are partial and 
% incremental, $m$-target rectifiability cannot be easily
% predetermined. Rectification convergence is also dependent upon the
% order in which the targets are selected for iterative patching. 
%  This offers
% better scalability, as well enables the selection of a minimum number
% ($m$) of rectification targets for a given buggy circuit.

%and also computes $m$-rectification functions in one-go.
%%  Despite these breakthroughs, the problem of
%% rectification of datapath circuits is still unsolved, hence a scalable
%% solution to compute and synthesize subfunctions in arithmetic circuits
%% is still desired. 

%% Recently, an efficient approach on resource aware ECO patch 
%% generation has been presented in~. A multi-fix approach
%% with heuristics to select effective rectification targets has been presented in
%% \cite{SS_Fujita:ISCAS19}. However, these techniques are also SAT and/or
%% Craig-interpolation based, which are known to be infeasible for arithmetic
%% circuits with large operand widths. 

%% A robust ECO approach to derive patches with minimal 
%% impact on the heavily optimized existing $\impl$ against 
%% a structurally dissimilar ECO-evolved $\spec$ has been presented in
%% . The authors propose enumerating rectification points 
%% functionally and match the circuitry of patches implicitly to maximize reuse of 
%% existing logic in the $\impl$. To achieve scalability, the method proposes 
%% modeling and analyzing Boolean reasoning queries in symbolic sampling domain. 
%% The signal selection heuristic (main contribution) proposed in \cite{SS_Roland:DAC19}
%% is orthogonal to the setup and approach presented in our paper.
%% However, the authors don't discuss the application of the proposed technique to 
%% arithmetic circuits, hence an efficient/scalable solution is still desired. 

{\it Problem Statement:}
We are given the following: i) as the $\spec$, a multivariate
polynomial $f$ with coefficients in a finite field of $2^n$ elements
(denoted $\F_{2^n}$), for a given  $n\in \Z_{\geq 1}$; ii) a primitive
polynomial $P_n(X)$ of degree $n$ with coefficients in $\{0,1\}$
to construct $\Fkn$; iii) an incorrect $\impl$ circuit $C$,
with no assumptions on the number or the type of bugs present in
$C$; and
% We perform formal verification of $C$ against $f$ (using \cite{lv:tcad2013})
% to detect the presence of bugs, and determine
% that $C$ is not single-fix rectifiable~\cite{Vkrao:FMCAD18}. 
iv) a set of $m$ targets from $C$, provided beforehand or selected using the heuristics 
proposed in~\cite{SS_Alan:DAC18,SS_Fujita:ISCAS19,SS_Roland:DAC19}.
%It is already ascertained by verification that 
%$C$ does not match the \spec, and it is further determined that $C$ is
%also not {\it single-fix rectifiable} w.r.t. $f$. The approaches of
%\cite{Utkarsh:VLSI18, Vkrao:FMCAD18} are therefore inapplicable; hence
%MFR checking of $C$ is required. % so that $C$ matches $f$. 
The circuit may or may not be rectifiable at these $m$ targets.
{\it The objective is to check whether the given set of $m$
  targets collectively admit multi-fix rectification.} 
 This MFR-check ascertains whether rectification functions exist that can
patch $C$ at these $m$-targets. 
%ii) If $C$ admits MFR, then compute $m$-bit rectification
%functions, along with corresponding don't care conditions, to
%synthesize these functions and patch the circuit. 

% {\it Approach and Contributions:} 
%We formulate MFR using symbolic computer algebra over finite fields. 
%Given a $\spec$ polynomial $f$ over a finite field of $2^n$ elements ($\Fkn$)
{\it Approach:}
The given $\impl$ $C$, with operand word-length $n$, is modeled
as a polynomial ideal in the multivariate polynomial ring with
coefficients in the finite field $\Fkn$, denoted $\Fkn[x_1,\dots,x_d]$. 
The $m$ targets are
collected as an $m$-bit-vector word $W$, which evaluates in $\Fkm$.
For the 
rectification check, the %hitherto 
unknown rectification function ($U$) is  modeled as a (word-level)
polynomial function in primary inputs (i.e. $W = U(X_{PI})$), which maps 
$n$-bit primary inputs $X_{PI}$ to an $m$-bit word $W :\F_2^{|X_{PI}|} \rightarrow \Fkm$.
The rectifiability check 
%and the computation of $W = U(X_{PI})$
is then formulated with algebraic geometry over finite fields 
%\cite{ideals:book}
%\cite{gao:gf-gb-ms},
and solved using \Grobner basis (GB) techniques~\cite{gb_book}.

% We present techniques that ascertain whether these $m$ targets
% collectively admit multi-fix rectification. 
% and if not, update our set of $m$ targets. 
%If so, we compute $m$-Boolean functions at the targets which admit
%rectification.
% Otherwise, the set of $m$ targets is updated and the process is
% repeated. 

{\it Contributions:}
% Our algebraic approach provides new insight into MFR checking
% of finite field arithmetic circuits.
% We propose new knowledge on application of algebraic approach
% to enable MFR checking of finite field arithmetic circuits. The approach
%  attains efficiency by operating at the word-level.
Our word-level algebraic approach enables efficient MFR checking 
of finite field arithmetic circuits.
% Our word-level algebraic approach
% enables MFR checking of buggy finite field arithmetic circuits.
% by efficiently operating at the word-level.
This word-level formulation poses new mathematical challenges: 
the field $\Fkm$ might not be compatible with the field
$\Fkn$, which prevents us from performing algebraic operations in a unified domain.
We overcome this problem by computing the smallest single field $\Fkk$
containing both $\Fkm, \Fkn$. This requires the computation of  
a specific primitive polynomial $P_k(X)$ for $\Fkk$. 
% which allows us to perform algebraic operations in a unified framework. 
We present an approach to compute
$P_k(X)$ using polynomial factorization and composite
fields. We utilize the unified framework and derive an efficient
GB-based word-level decision procedure for $m$-target MFR checking.  
Experiments conducted on various finite field
benchmarks with different operand and patch word-lengths, $n$ and $m$,
respectively, corroborate the efficacy of our approach.

% {\it Contribution 1:} Our algebraic approach
% enables MFR of finite field arithmetic circuits, and it
% efficiently operates at the word-level. The $m$ targets are
% collected as an $m$-bit-vector word $W$, which is then construed as a
% variable that takes values  in the finite field $\Fkm$. For the 
% rectification check, the %hitherto 
% unknown rectification function ($U$) is  modeled as a (word-level)
% polynomial function in primary inputs (i.e. $W = U(X_{PI})$), as a
% mapping $f_W:\F_2^{|X_{PI}|} \rightarrow \Fkm$, from $n$-bit primary
% inputs $X_{PI}$ to a $m$-bit word ($W$).

% {\it Contribution 2:} The rectifiability check 
% %and the computation of $W = U(X_{PI})$
% is formulated in algebraic geometry over finite fields 
% %\cite{ideals:book}
% %\cite{gao:gf-gb-ms},
% and solved using \Grobner basis (GB) techniques~\cite{gb_book}. 
% %% In this regard, our approach is conceptually inspired by the recent
% %% work of~\cite{Vkrao:FMCAD18}.
% The word-level formulation poses new mathematical challenges: 
% The field $\Fkm$ might not be compatible with (contained in) field
% $\Fkn$. We solve this problem over the smallest single field $\Fkk$
% containing both $\Fkm, \Fkn$. This requires the computation of 
% a specific primitive polynomial $P_k(X)$ for $\Fkk$ which allows
% to operate in a unified framework. We present an approach to compute
% $P_k(X)$ using polynomial factorization and composite
% fields~\cite{cfmulti:1996}. Using $P_k(X)$, we propose an efficient
% GB-based word-level decision procedure for $m$-target MFR check.  

%% The paper also presents 
%% pragmatic techniques to simplify the unified framework setup
%% to overcome the complexity of polynomial factorization. 
%{\it Contribution 3:}
%In addition, we present techniques to compute
%observability don't care (ODC) conditions corresponding to the $m$
%rectification functions. We show how in our algebraic model the ODC's
%correspond to {\it varieties of polynomial ideals}, and how they can
%be computed with \Grobner bases. 

%% a polynomial 
%% which can enumerate all the ODC points 
%% for the rectified targets. Overall, the heuristics 
%% to select initial candidate targets, modeling of bit-level targets 
%% as a word-level patch, the setting-up of a unified framework 
%% for MFR, and the concept of ODCs in MFR setup are the new contributions 
%% that haven't been presented in prior literature. 

%% Taking inspiration from~\cite{Utkarsh:TCAD19}, we utilize 
%% PolyBori’s~\cite{pbori:JSC09} reduction procedure with 
%% ZDDs~\cite{Minato:DAC93,Minato:DAC94} as the underlying data structure 
%% to improve our proposed approach.

% Our approach is implemented using the polynomial algebra computational
% engines of {\sc Singular} \cite{DGPS_410} and {\sc PolyBori}
% \cite{pbori:JSC09}. 



%% However, the feasibility of 
%% synthesizing any computed patch function is constrained by the cost 
%% defined as part of the design metrics. The quality of a patch function 
%% is heavily dependent on the effective selection of the rectification targets, 
%% reuse of the existing logic, and the amount of resultant design modifications.
%% % Since we select targets heuristically and our patch function is computed in terms 
%% % of primary inputs rather than the intermediate signals, we make no claims on the
%% % quality of the patch function computed. 
%% Exploiting symbolic algebra techniques
%% on generating low-cost patch functions with effective signal selection heuristics 
%% needs further investigation and tuning, which is beyond the scope of this paper.

% {\it Paper Organization:} The following section covers preliminary
% background. Section~\ref{sec:verify} reviews verification formulation,
% the result of which is analyzed to prune rectification
% targets. Section~\ref{sec:comps} describes heuristics to pick targets 
% and how the unified framework is setup. Word-level rectification check 
% is described in Section~\ref{sec:rcheck}, followed by a brief description 
% in Section~\ref{sec:rfunc} on rectification function computation 
% (Analogous to that of Section VI in~\cite{Vkrao:FMCAD18}). Results are 
% presented in Section \ref{sec:exp} and Section~\ref{sec:conc} concludes the paper.

{\it Paper Organization:} The following section covers preliminary
background. Section~\ref{sec:pmodel} reviews the polynomial modeling concepts.
% an augmentation of which is later utilized in MFR checking.
Section~\ref{sec:comps} describes the setup of a unified computational
framework for MFR checking. Word-level rectification check is described in
Section~\ref{sec:rcheck}.
%, followed by  rectification function computation and ODC concept in
%Section~\ref{sec:rfunc}.
Experimental results are described in Section \ref{sec:exp}, and
Section~\ref{sec:conc} concludes the paper. 